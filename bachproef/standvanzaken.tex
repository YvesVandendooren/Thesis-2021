\chapter{\IfLanguageName{dutch}{Stand van zaken}{State of the art}}
\label{ch:stand-van-zaken}

Zoals eerder vermeld focust dit onderzoek zich op Security Information and Event Management en hoe dit kan geïmplementeert en gebruikt worden binnen IT-omgevingen.
Deze sectie geeft een beeld van hoe SIEM ontstaan is en hoe een SIEM tool werkt. Daarnaast werd er ook gekeken naar de verschillende providers van dergelijke tooling, alsook op wat er moet gelet worden voor de implementatie.

\section{\IfLanguageName{dutch}{Geschiedenis}{Problem Statement}}

De term Security Information and Event Management, bedacht door Mark Nicolett en Amrit Williams in 2005, beschrijft de mogelijkheden om informatie te verzamelen, analyseren en presenteren [Eva Kostrecova]. SIEM is een combinatie van Security Information Mangement (SIM) en Security Event Management (SEM) functionaliteiten in één security management systeem. 
SIM focust vooral op het verzamelen en analyseren van historische data bedoelt voor het verbeteren van opslag prestatie in IT infrastructuur. Daartegenover staande is SEM, dat voornaamlijke focust op aggregatie van data in een overzichtelijk formaat dat helpt bij het afhandelen van security incidenten. [Morteza Zeinali]  
\section{\IfLanguageName{dutch}{Wat is SIEM}{Problem Statement}}

\section{\IfLanguageName{dutch}{Providers}{Problem Statement}}

\section{\IfLanguageName{dutch}{SIEM implementatie}{Problem Statement}}

% Tip: Begin elk hoofdstuk met een paragraaf inleiding die beschrijft hoe
% dit hoofdstuk past binnen het geheel van de bachelorproef. Geef in het
% bijzonder aan wat de link is met het vorige en volgende hoofdstuk.

% Pas na deze inleidende paragraaf komt de eerste sectiehoofding.

Dit hoofdstuk bevat je literatuurstudie. De inhoud gaat verder op de inleiding, maar zal het onderwerp van de bachelorproef *diepgaand* uitspitten. De bedoeling is dat de lezer na lezing van dit hoofdstuk helemaal op de hoogte is van de huidige stand van zaken (state-of-the-art) in het onderzoeksdomein. Iemand die niet vertrouwd is met het onderwerp, weet nu voldoende om de rest van het verhaal te kunnen volgen, zonder dat die er nog andere informatie moet over opzoeken \autocite{Pollefliet2011}.

Je verwijst bij elke bewering die je doet, vakterm die je introduceert, enz. naar je bronnen. In \LaTeX{} kan dat met het commando \texttt{$\backslash${textcite\{\}}} of \texttt{$\backslash${autocite\{\}}}. Als argument van het commando geef je de ``sleutel'' van een ``record'' in een bibliografische databank in het Bib\LaTeX{}-formaat (een tekstbestand). Als je expliciet naar de auteur verwijst in de zin, gebruik je \texttt{$\backslash${}textcite\{\}}.
Soms wil je de auteur niet expliciet vernoemen, dan gebruik je \texttt{$\backslash${}autocite\{\}}. In de volgende paragraaf een voorbeeld van elk.

\textcite{Knuth1998} schreef een van de standaardwerken over sorteer- en zoekalgoritmen. Experten zijn het erover eens dat cloud computing een interessante opportuniteit vormen, zowel voor gebruikers als voor dienstverleners op vlak van informatietechnologie~\autocite{Creeger2009}.

\lipsum[7-20]
